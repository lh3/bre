\documentclass[10pt]{article}
\usepackage{color}
\definecolor{gray}{rgb}{0.7,0.7,0.7}
\usepackage{framed}
\usepackage{enumitem}
\usepackage{longtable}

\usepackage[margin=1.0in,footskip=0.25in]{geometry}

\renewcommand{\arraystretch}{1.2}
\renewcommand{\ttdefault}{cmtt}

\title{BRE: a Binary Run-length Encoding format}
\author{}
\date{\vspace{-8ex}}

\begin{document}
\maketitle

{\small
BWT construction at the terabyte scale is resource demanding.
Due to the lack of an exchange format, most BWT-based tools construct BWT from scratch and cannot use prebuilt BWTs.
Even when they use the same BWT construction algorithm, they may map symbols to intergers differently.
This makes it difficult for BWT-based tools work together.

BRE is a binary format to store strings in the run-length encoding.
It introduces an interface between BWT-based tools such that a BWT constructed with one tool can be relatively easily used by other tools.
BRE aims to be simple such that it can be read and write easily.
It may not be space efficient and does not support advanced operations such as rank/select.

The following table describes the binary BRE format.
All integers are represented in little endian.
}
\begin{table}[ht]
\centering
{\small
\begin{tabular}{|l|l|p{6.7cm}|l|l|}
  \cline{1-5}
  \multicolumn{2}{|c|}{\bf Field} & \multicolumn{1}{c|}{\bf Description} & \multicolumn{1}{c|}{\bf Type} & \multicolumn{1}{c|}{\bf Value} \\\cline{1-5}
  \multicolumn{2}{|l|}{\sf magic} & BRE magic string & {\tt char[4]} & {\tt BRE\char92 1}\\\cline{1-5}
  \multicolumn{2}{|l|}{\sf b\_per\_sym} & Number of bytes per symbol & {\tt uint8\_t} & $[1,8]$ \\\cline{1-5}
  \multicolumn{2}{|l|}{\sf b\_per\_run} & Number of bytes per run & {\tt uint8\_t} & $[0,8]$ \\\cline{1-5}
  \multicolumn{2}{|l|}{\sf atype} & Predefined alphabet type. 0 for undefined. & {\tt uint8\_t} & $[0,256)$ \\\cline{1-5}
  \multicolumn{2}{|l|}{\sf mtype} & Type of multi-string BWT & {\tt uint8\_t} & $[0,256)$ \\\cline{1-5}
  \multicolumn{2}{|l|}{\sf asize} & Size of the alphabet \emph{including} the sentinel & {\tt int64\_t} & $[2,2^{63})$ \\\cline{1-5}
%  \multicolumn{2}{|l|}{\sf n\_rec} & Number of records. $-$1 for missing data. & {\tt int64\_t} & $[-1,2^{63})$ \\\cline{1-5}
  \multicolumn{2}{|l|}{\sf l\_aux} & Length of auxiliary data in bytes & {\tt int64\_t} & $[0,2^{63})$ \\\cline{1-5}
  \multicolumn{2}{|l|}{\sf aux} & Auxiliary data & {\tt uint8\_t[l\_aux]} & \\\cline{1-5}
  \multicolumn{5}{|c|}{\textcolor{gray}{\it List of records until (0,0) is read}} \\\cline{2-5}
  & {\sf sym} & Symbol & {\tt uint8\_t[b\_per\_sym]} & $[0,{\sf asize})$ \\\cline{2-5}
  & {\sf run\_len} & Run length. A long run may be split. & {\tt uint8\_t[b\_per\_run]} & $[1,2^{8\cdot{\sf b\_per\_run}})$ \\\cline{1-5}
  \multicolumn{2}{|l|}{\sf eof\_sym} & End-of-file symbol & {\tt uint8\_t[b\_per\_sym]} & 0 \\\cline{1-5}
  \multicolumn{2}{|l|}{\sf eof\_len} & End-of-file run length & {\tt uint8\_t[b\_per\_run]} & 0 \\\cline{1-5}
\end{tabular}}
\end{table}

{\small
Sentinels shall be mapped to integer 0.
Predefined alphabet ``{\sf atype}'' may take following values:
\begin{enumerate}
\item {\sf ASCII} (${\sf asize}=128$)
\item {\sf DNA6} (${\sf asize}=6$). {\tt \$}/{\tt A}/{\tt C}/{\tt G}/{\tt T}/{\tt N} is mapped to integer 0--5, respectively.
\item {\sf DNA16} (${\sf asize}=16$). {\tt \$ACMGRSVTWYHKDBN} is mapped to 0--15, respectively.
\end{enumerate}

The type of multi-string BWT ``{\sf mtype}'' can be:
\begin{enumerate}
\setcounter{enumi}{-1}
\item Undefined
\item Single string. This includes multi-string concatenation without sentinels, or concatenation with a special symbol.
  There should be exactly one sentinel.
\item Multi-string BWT, simple concatenation. Sentinels are not distinguished from each other.
\item Multi-string BWT, input order. Sentinels ordered by input.
\item Multi-string BWT, other order.
\end{enumerate}

Potential caveats and alternatives:
\begin{itemize}
\item Many tools require suffix array samples for the ``locate'' operation.
  We need to provide a tool to compute suffix array samples from BWT.
\item It is more space efficient to bit-pack {\sf sym} and {\sf run\_len} but this will complicate reading and writing.
\item A BRE reader would prefer to see the number of symbols and number of runs in the header.
  However, a BRE writer may not have this information at the beginning.
  This may happen, for example, when we read a BWT in plain text from the standard input and convert it to BRE.
  For a similar reason, {\sf n\_rec} is not always easy to obtain and thus is allowed to be missing.
\end{itemize}
}

\end{document}
